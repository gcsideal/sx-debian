\chapter{Client operations}

\section{Access profiles}
\label{sec:profiles}
Using \path{sxinit} one can configure access for multiple users
and clusters. The access profiles have a format of
\path{sx://[PROFILE_NAME@]cluster_name}, where \path{PROFILE_NAME}
is an arbitrary name for the profile, and doesn't need to match
the username for which the access is configured. In case of a single
user access to a cluster, \path{PROFILE_NAME} can be omitted and
\path{sx://cluster_name} will be the default profile for a given
cluster.

\subsection{Adding profiles}
To add an access profile for the user 'jeff' and the local cluster
created in previous chapters run the following command:
\begin{lstlisting}
$ sxinit -l 192.168.1.101 -A @jeff sx://jeff@mycluster
Warning: self-signed certificate:

        Subject: C=GB, ST=UK, O=SX, CN=mycluster
	Issuer: C=GB, ST=UK, O=SX, CN=mycluster
	SHA1 Fingerprint: 84:EF:39:80:1E:28:9C:4A:C8:80:E6:56:57:A4:CD:64:2E:23:99:7A

Do you trust this SSL certificate? [y/N] ^\marked{y}^
Trusting self-signed certificate
Please enter the user key: ^\marked{FqmlTd9CWZUuPBGMdjE46DaT1/3kx+EYbahlrhcdVpy/9ePfrtWCIgAA}^
\end{lstlisting}
Since "mycluster" is not a DNS name, we had to point \path{sxinit} 
to one of the nodes of the cluster. That allowed it to connect and
discover all the other nodes. We also created the alias \path{@jeff},
which will be used for convenience.

\subsection{Listing access profiles}
To list all configured access profiles run:
\begin{lstlisting}
$ sxinit --list
sx://jeff@mycluster                       @jeff
sx://admin@mycluster                      @cluster
\end{lstlisting}

\subsection{Deleting profiles}
To delete a profile run the following command and provide a full URI
or an alias name as follows:
\begin{lstlisting}
$ sxinit --delete @somealias
\end{lstlisting}
It will delete the alias \path{@somealias} and the profile associated with
it.

\section{Working with files} \label{sec:files}
\SX provides easy to use file tools, which resemble typical UNIX commands.
Since \SX is an object storage and not a filesystem, there are some
fundamental differences, though. One of them is lack of "real" directories:
each file (object) has assigned a full path that uniquely identifies it and
the path is not a part of any tree structure. \SX does simulate a directory
structure by matching the subpaths, for example \path{/path/file1}
and \path{/path/file2} will be presented as contents of the directory
\path{/path/} just like on a typical filesystem. However, the directory
\path{/path/} is only emulated (and is not assigned to any object), and
therefore it's perfectly legit to also have a file with a path \path{/path},
which doesn't conflict with the other two files at all! 

In the following subsections we present the command line tools and show
how to use them to perform common tasks.

\subsection{sxcp: upload and download files}
\path{sxcp} can copy files and entire directories from and to Skylable \SX
clusters. It can also copy data between two different SX clusters. By default,
for each file a progress bar is displayed, which shows the transfer speed and
the estimated time of arrival. \path{sxcp} makes use of all the advanced features \SX,
such as deduplication and transfer resuming to minimize the bandwidth usage.

Use \path{sxcp -r} to recursively upload directories to the remote volume:
\begin{lstlisting}
$ sxcp -r /home/jeff/VMs/ @jeff/vol-jeff/VMimages/
Uploading /home/jeff/VMs/FreeBSD 10.0/FreeBSD 10.0.vmdk (size: 4.91GB)
  14% [====>                                            ]  55.11MB/s ETA 44s
\end{lstlisting}
\path{sxcp} shows the average speed of the tranfer and how long it will take.
The great feature of \SX is the already mentioned transfer resuming, which
allows to continue the transfer in case it was interrupted. Below we interrupt
the transfer of the large file and repeat the same copy again:
\begin{lstlisting}
$ sxcp -r /home/jeff/VMs/ @jeff/vol-jeff/VMimages/
Uploading /home/jeff/VMs/FreeBSD 10.0/FreeBSD 10.0.vmdk (size: 4.91GB)
  94% [=============================================>   ]  55.11MB/s ETA 5s
^\textasciicircum^CProcess interrupted
$ sxcp -r /home/jeff/VMs/ @jeff/vol-jeff/VMimages/
  97% [++++++++++++++++++++++++++++++++++++++++++++++=> ]  52.17MB/s ETA 2s
\end{lstlisting}
The second \path{sxcp} call automatically finds out, which blocks of the file
has already been transferred and only uploads the missing ones. The transfer
resuming works in a similar way for file downloads.

\path{sxcp} can copy files between different volumes, also on different
clusters, and comes with other useful features, such as bandwidth limiting. See
\path{man sxcp} for the usage details and other examples.

\subsection{sxls: list volumes and files}
With \path{sxls} one can discover, which volumes are accessible on the cluster
and then list their contents. To get the list of volumes, which user 'jeff'
can access run:
\begin{lstlisting}
$ sxls -lH @jeff
  VOL  rep:2  rev:1  rw  -  12.83G  50.00G  25% sx://jeff@mycluster/vol-jeff
\end{lstlisting}
With \path{-l (--long)} and \path{-H (--human-readable)} options \path{sxls}
displays the list of available volumes, together with additional information
such as the replica count, maximum number of revisions per file, permissions,
size and usage.

Running \path{sxls} against the volume without any arguments returns the first
level of files, similarly to the command \path{ls}:
\begin{lstlisting}
$ sxls @jeff/vol-jeff
sx://jeff@mycluster/vol-jeff/VMimages/
\end{lstlisting}
To list the volume recursively, with more information about files and human
readable sizes run:
\begin{lstlisting}
$ sxls -rlH @jeff/vol-jeff
2014-11-17 14:03           31 sx://jeff@mycluster/vol-jeff/VMimages/Debian-MIPS/bridge.sh
2014-11-17 14:03      245.88M sx://jeff@mycluster/vol-jeff/VMimages/Debian-MIPS/debian_squeeze_mips_standard.qcow2
2014-11-17 14:03        4.10M sx://jeff@mycluster/vol-jeff/VMimages/Debian-MIPS/initrd.gz
2014-11-17 14:03          139 sx://jeff@mycluster/vol-jeff/VMimages/Debian-MIPS/run
2014-11-17 14:03          677 sx://jeff@mycluster/vol-jeff/VMimages/Debian-MIPS/start.sh
2014-11-17 14:03        6.61M sx://jeff@mycluster/vol-jeff/VMimages/Debian-MIPS/vmlinux-2.6.32-5-4kc-malta
2014-11-17 14:03        1.41G sx://jeff@mycluster/vol-jeff/VMimages/Debian-PPC/debian_squeeze_powerpc_standard.qcow2
2014-11-17 14:04          349 sx://jeff@mycluster/vol-jeff/VMimages/Debian-PPC/start.sh
2014-11-17 14:02        4.91G sx://jeff@mycluster/vol-jeff/VMimages/FreeBSD 10.0/FreeBSD 10.0.vmdk
2014-11-17 14:03      693.12M sx://jeff@mycluster/vol-jeff/VMimages/FreeBSD 10.0/FreeBSD-10.0-BETA1-amd64-disc1.iso
\end{lstlisting}

\subsection{sxmv: move or rename files}
\path{sxmv} can move files or group of files into new locations. It can be
used to just rename individual files or move entire groups to another cluster.
In contrast to the command \path{mv}, renaming a directory with \path{sxmv}
requires prividing the recursive flag \path{-r}. That's because of the design
of the object storage and lack of real directories as described at the beginning
of this chapter. In order to rename a directory, \path{sxmv} has to rename all the
files (objects), which share the same directory path. In the example below we rename
the directory 'VMimages' to 'VMs' and list the new volume structure in basic mode:
\begin{lstlisting}
$ sxmv -r @jeff/vol-jeff/VMimages/ @jeff/vol-jeff/VMs/
$ sxls -r @jeff/vol-jeff
sx://jeff@mycluster/vol-jeff/VMs/Debian-MIPS/bridge.sh
sx://jeff@mycluster/vol-jeff/VMs/Debian-MIPS/debian_squeeze_mips_standard.qcow2
sx://jeff@mycluster/vol-jeff/VMs/Debian-MIPS/initrd.gz
sx://jeff@mycluster/vol-jeff/VMs/Debian-MIPS/run
sx://jeff@mycluster/vol-jeff/VMs/Debian-MIPS/start.sh
sx://jeff@mycluster/vol-jeff/VMs/Debian-MIPS/vmlinux-2.6.32-5-4kc-malta
sx://jeff@mycluster/vol-jeff/VMs/Debian-PPC/debian_squeeze_powerpc_standard.qcow2
sx://jeff@mycluster/vol-jeff/VMs/Debian-PPC/start.sh
sx://jeff@mycluster/vol-jeff/VMs/FreeBSD 10.0/FreeBSD 10.0.vmdk
sx://jeff@mycluster/vol-jeff/VMs/FreeBSD 10.0/FreeBSD-10.0-BETA1-amd64-disc1.iso
\end{lstlisting}

\subsection{sxrm: remove files}
The equivalent of the system command \path{rm} in \SX is \path{sxrm}. Similarly
to other tools, it can handle individual files or entire directories in
recursive mode. Below we first check the current space usage for the volume, then
remove a directory with some large files (using a wildcard to match it), and check
the usage again:
\begin{lstlisting}
$ sxls -l @jeff
  VOL  rep:2  rev:1  rw  -  7.24G   50.00G  14% sx://jeff@mycluster/vol-jeff
$ sxrm -r @jeff/vol-jeff/VMs/FreeBSD*
Deleted 2 file(s)
$ sxls -lH @jeff
  VOL  rep:2  rev:1  rw  -  1.66G   50.00G  14% sx://jeff@mycluster/vol-jeff
\end{lstlisting}

\subsection{sxrev: manage file revisions}
The \SX volumes can be configured \footnote{See \fref{sec:volumes} on how
to create and configure volumes} to keep multiple revisions of files. For
example, if a volume was created with an option to keep 3 revisions, every
time a specific file gets modified the previous copy will be preserved and
the latest 3 versions of the file will be available for download. The tools
such as \path{sxcp} or \path{sxls} will always operate on the latest revision.
In order to access and manage the older revisions, one has to use \path{sxrev}.

In the examples below we will operate on the volume \path{vol-jeff-rev}, which
was configured to store up to 3 revisions for each file and the example file
\path{document.pdf} was already updated a few times. In order to list all of
its revisions, run the following command:
\begin{lstlisting}
$ sxrev list @jeff/vol-jeff-rev/document.pdf
Revisions for file @jeff/vol-jeff-rev/document.pdf (most recent first):
1.      2014-11-18 12:05 size:128026 rev:"2014-11-18 12:05:00.938:d2bc1190a0f70f4b4925d702e0d567a7"
2.      2014-11-18 11:54 size:105866 rev:"2014-11-18 11:54:42.362:1fc102f66cabd0e8daac8e1279b54c0a"
3.      2014-11-18 10:23 size:93545  rev:"2014-11-18 10:23:22.188:d3b1fb1d7e4219ab4a8d1fc7c8edff0c"
\end{lstlisting}
The first revision on the list is the latest one, which is also visible to
other tools. In order to restore an older revision of a file, it needs to
be copied into a new destination. By default \path{sxrev} asks interactively,
which revision should be copied as on the example below:
\begin{lstlisting}
$ sxrev copy @jeff/vol-jeff-rev/document.pdf ~/document-prev.pdf
Revisions for file @jeff/vol-jeff-rev/document.pdf (most recent first):
1.      2014-11-18 12:05 size:128026 rev:"2014-11-18 12:05:00.938:d2bc1190a0f70f4b4925d702e0d567a7"
2.      2014-11-18 11:54 size:105866 rev:"2014-11-18 11:54:42.362:1fc102f66cabd0e8daac8e1279b54c0a"
3.      2014-11-18 10:23 size:93545  rev:"2014-11-18 10:23:22.188:d3b1fb1d7e4219ab4a8d1fc7c8edff0c"
Choose revision to copy: 2
Copy operation completed successfully
\end{lstlisting}
The same operation can be performed in non-interactive mode by providing
the revision string as an argument:
\begin{lstlisting}
$ sxrev copy -r "2014-11-18 11:54:42.362:1fc102f66cabd0e8daac8e1279b54c0a" @jeff/vol-jeff-rev/document.pdf ~/document-prev.pdf
Copy operation completed successfully
\end{lstlisting}
The size of all revisions adds up to the volume usage, that's why one may want
to remove specific revisions (eg. for large media files). When a file with
multiple revisions gets deleted with \path{sxrm}, all of the revisions get
removed automatically as well. With \path{sxrev delete} only specific revisions
can be deleted and it works similarly to \path{sxrev copy}. In the example
below we remove the two oldest revisions, in both interactive and
non-interactive modes:
\begin{lstlisting}
$ sxrev delete @jeff/vol-jeff-rev/document.pdf
Revisions for file @jeff/vol-jeff-rev/document.pdf (most recent first):
1.      2014-11-18 12:05 size:128026 rev:"2014-11-18 12:05:00.938:d2bc1190a0f70f4b4925d702e0d567a7"
2.      2014-11-18 11:54 size:105866 rev:"2014-11-18 11:54:42.362:1fc102f66cabd0e8daac8e1279b54c0a"
3.      2014-11-18 10:23 size:93545  rev:"2014-11-18 10:23:22.188:d3b1fb1d7e4219ab4a8d1fc7c8edff0c"
Choose revision to delete: 2
Delete operation completed successfully
$ sxrev delete -r "2014-11-18 10:23:22.188:d3b1fb1d7e4219ab4a8d1fc7c8edff0c" @jeff/vol-jeff-rev/document.pdf
Delete operation completed successfully
$ sxrev list @jeff/vol-jeff-rev/document.pdf
Revisions for file @jeff/vol-jeff-rev/document.pdf (most recent first):
1.      2014-11-18 12:05 size:128026 rev:"2014-11-18 12:05:00.938:d2bc1190a0f70f4b4925d702e0d567a7"
\end{lstlisting}
